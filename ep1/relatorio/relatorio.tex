\documentclass[a4paper]{article}
\usepackage[T1]{fontenc}
\usepackage[utf8]{inputenc}
\usepackage{lmodern}
\usepackage[brazil]{babel}
\usepackage{amsmath}
\usepackage{graphicx}
\usepackage{listings}
\usepackage[colorinlistoftodos]{todonotes}

\title{Relatório Exercício Programa I - Batalha de Robôs}
\author{
Fellipe Souto Sampaio \footnote{Número USP: 7990422 e-mail: fellipe.sampaio@usp.com}\\
Gervásio Protásio dos Santos Neto \footnote{Número USP: 7990996 e-mail: gervasio.neto@usp.br}\\
Vinícius Jorge Vendramini \footnote{Número USP: 7991103 e-mail: vinicius.vendramini@usp.br}
}

\begin{document}
\maketitle

\begin{center}
MAC 0242 Laboratório de Programa\c{c}ão II \\
Prof. Marco Dimas Gubitoso \\
             
\end{center}

\begin{center}
Instituto de Matemática e Estatística - IME USP \\
 Rua do Matão 1010 \\
 05311-970\, Cidade Universitária, São Paulo - SP \\
\end{center}

\newpage

\section{Introdução}
Este relatório pretende explicar a implementação do exercício programa I detalhando seu funcionamento, o projeto de suas classes e seus objetos.\\\\
Detalharemos a forma de implementação e manuseio da pilha e sua relação com o módulo operação, e o uso das expressões regulares para o processamento da entrada de dados.

\section{Expressões Regulares}

Para processamento do arquivo-fonte contendo o prorama a ser interpretado foram utilizadas expressões regulares do Perl.\\\\
Usou-se o operador diamante (<>) dentro de um loop while para ler a entrada até o final e a cada linha da entrada viu-se se ela se conformava com o esperado de uma linha de código.\\\\
Mais especificamente, a seguinte expressões regulares aparecem em nossa implementação do processador de texto:
\begin{lstlisting}[frame=single]
 /^#.*[\n\f]*/ #verifica se se trata de um comentario
 
 /(\b[a-zA-Z]*\b:\s*)[\n\f#]*$/ #identifica uma linha
				# so com label
                
 /(\b[a-zA-Z]*\b:\s*)?(\b[a-zA-Z]{2,4}\b[^:]?)
 (\w*)\s*[\n\f#]*/ #identifica uma linha executavel
 

\end{lstlisting}
Durante o processamento do textos, dependendo do que foi capturado pela expressão regular, realiza-se um comando:

\begin{itemize}
\item Comentário: Linhas de comentário e comentários em linhas válidas são ignorados.
\item Label: caso a label seja uma string válida (não é um undef), ela é inserida em um hash na stack (uma instância da classe pilha, detalhado mais a frente). A label é usada como key do hash, o valor guardado é a posição do programa marcada pelo label.
\item Comandos: Ao identificar um par ordenado (comando,argumento) é criada uma referência anônima para esse par que é então inserido no vetor de instruções.
\end{itemize}

\section{Vetor de Instruções}

O vetor de instruções é uma estrura de dados que contém a sequência dos comandos que devem ser executados, acompanhados de seus respectivos argumentos.\\\\
Sempre que é encontrada uma instrução válida, ela e seu argumentos são inseridos em um vetor anônimo, e este então é inserido no vetor de instruções.\\\\
Durante o laço de execução das instruções o vetor é percorrido e seu conteúdo (os comandos) são interpretardos pelo método \textit{makeOperation}, pretencente a classe pilha. \\\\
Com execessão de quando é executado um pulo (JMP), a leitura do vetor é linear, como a sua posição atual (o Program Counter) sendo um atributo da pilha.

\section{Pilha de Dados}
Na implementação de uma máquina virtual o elemento principal é a pilha de dados. Em nosso programa foi criado a classe pilha, um objeto munido de operações usuáis, como empilhar, desempilhar, devolver o topo, duplicar entre outros. Todas estas operações são aplicadas diretamente na pilha, independente do tipo de dado empilhado.\\\\
O método principal de operação da pilha é chamado \textit{makeOperation}, no qual a instrução a ser executada é recebida, testada se é válida e se está presente no hashing de instruções, em caso positivo o conteudo é uma referência para qual médoto deve ser aplicado (saltos, empilhamento, comparação lógica do conteúdo, descarte, impressão ...). No segundo caso, negativo, é verificado se a instrução é chave do hashing do objeto operation.

\section{Operações Lógicas e Aritméticas}
Neste módulo está a classe operation, responsável pelas operações lógicas e aritméticas entre elementos da pilha. Caso a operação requisitada exista esta é aplicada aos elementos que foram desempilhados da pilha. O valor final da operação binária é devolvido para que seja empilhado. Semelhante a implementação da classe stack,o hashing das operações tem como conteúdo referências para as funções que devem ser aplicadas.

\section{Integração}
Na execução do projeto procurou-se ao máximo atender a um projeto de orientação a objeto.\\\\
Os módulos stack e operations regimentam a execução. Na classe pilha um de seus atributos é um objeto operation.\\\\
É no módulo stack que são chamadas (por meio de referências anônimas) as funcções responsáveis pela execução de operações. Primeiro checa-se se a operação a ser realizada é uma operação de pilha; se for, é realizada. Cado contrário a operação requisitada não é uma manipulação usual, e sim uma operação lógico-aritmética sobre seus elementos, e para isso invoca-se a função pertinente do operations por meio do objeto previamente instanciando.\\\\
Por fim o readSource executa um loop (um ciclo Fetch, Decode, Execute - FDX), chamando as funções da pilha que processam comandos por meio de um objeto to tipo stack. O loop rodará até o final  do vetor de instruções ou encontrar-se o comando END.\\\\


\section {\textbf{\text{\Large  Anexo - Códigos Testados}}}

\title{\large Sequencia de Fibonacci}
\begin{lstlisting}[frame=single]
PUSH 1
PUSH 0
STO 0
STO 1
PUSH 20
STO 2
LOOP: 
RCL 0
RCL 1
DUP
STO 0
ADD
DUP
STO 1
PRN
RCL 2
PUSH 1
SUB
DUP
STO 2
PUSH 0
EQ
JIF LOOP
END
\end{lstlisting}


\title{\large Potências de 2}
\begin{lstlisting}[frame=single]
# inicializa
PUSH 1
STO 0
PUSH 10000000
STO 1

LOOP: POP 
RCL 0
DUP
ADD
STO 0
RCL 0
PRN
RCL 0
RCL 1
LT
JIF	LOOP

END
\end{lstlisting}
\newpage
\title{\large A resposta para a pergunta fundamental da vida, universo e tudo mais}
\begin{lstlisting}[frame=single]
PUSH 10
PUSH 4
ADD
PUSH 3
MUL
PRN
END

\end{lstlisting}

\end{document}